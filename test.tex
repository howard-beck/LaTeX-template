\documentclass{amsart}

\usepackage{cmds/packages}

\makeatletter

\begin{document}
$$ \S\compl[\MU] $$
Imagine maps:
$$ \Map*^G \@left@paren A, B\@right@paren $$
$$ \Map*^G (A, B) $$
where $ A, B \in \Top*^G $ (where $ \Top $ is a suitable category).

If $ X \in \Top_* $ and $ Y \in \Top $, we can form the spectra:
$$ \Suspinf X, \Suspinf+ Y $$
and for a spectrum $ Z \in \Sp $, we can form the pointed space:
$$ \Loopinf Z $$

\begin{thm}[Pythagoras]
$ a^2 + b^2 = c^2 $.
\begin{proof}
Trivial
\end{proof}
\end{thm}

\begin{lemma}[Zorn's Lemma]
An poset $ P \in \Poset $ where all chains are bounded above has a maximal element.
\begin{proof}
Axiom of choice.
\end{proof}
\end{lemma}

\begin{prop}
$ \sqrt[3]{2} $ is not a rational number.
\begin{proof}
Suppose it was.
Then, there will be two integers $ a, b \in \Z $ such that $ a^3 + a^3 = b^3 $.
We will need the following:
\begin{thm}[Fermat's Last Theorem, due to Andrew Wiles]
The equation $ a^n + b^n = c^n $ has no integer solutions when $ n \ge 3 $.
\begin{proof}
Trivial
\end{proof}
\end{thm}
We see that $ \sqrt[3]{2} $ being rational produces a contradiction to Fermat's Last Theorem.
Therefore, it must be irrational.
\end{proof}
\end{prop}

\begin{rmk}[Jordan curve theorem]
It happens
\begin{proof}[idea]
Duh
\end{proof}
\end{rmk}

$$ \S, \MU, \tmf, \BP, \T(n), \K(n), \HZ, \LT_n $$
$$ S, MU, tmf, BP, T(n), K(n), H\Z, E_n $$

\begin{claim}
When $ z \in \C $, we have that:
$$ e^z = e^{\Re(z)} (\cos \Im(z) + i \sin \Im(z)) $$
\end{claim}

$$ \K(\Z) \quad K \quad \K \quad (n) \quad \K(n - \cork_p (\family)) \quad \K(n) \quad \K(\Z) \quad \K(\infty)$$

$$ \Delta\!\!\!\!\!\scalebox{0.8}{\mbox{$\Delta$}}$$

\begin{axiom*}[Axiom of Choice]
For any collection of non-empty sets $ \mathcal{S} $, there exists a choice function $ \phi: \mathcal{S} \to \bigunion_{s \in \mathcal{S}} s $, such that $ \phi(s) \in s $.
\end{axiom*}

abcdefghijklmnopqrstuvwxyz

ABCDEFGHIJKLMNOPQRSTUVWXYZ

$$ abcdefghijklmnopqrstuvwxyz $$
$$ ABCDEFGHIJKLMNOPQRSTUVWXYZ $$
$$ \alpha\beta\gamma\delta\epsilon\zeta\eta\theta\iota\kappa\lambda\mu\nu\xi\omicron\pi\rho\sigma\tau\upsilon\phi\chi\psi\omega $$
$$ \Alpha\Beta\Gamma\Delta\Epsilon\Zeta\Eta\Theta\Iota\Kappa\Lambda\Mu\Nu\Xi\Omicron\Pi\Rho\Sigma\Tau\Upsilon\Phi\Chi\Psi\Omega $$
$$ \varepsilon\vartheta\varrho\varphi $$

$$ \sum \prod \coprod \bigunion \bigdjunion \bigdsum \bigtprod $$

\end{document}